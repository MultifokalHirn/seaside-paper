% Koma-Script Basisklasse
\documentclass[a4paper,12pt,pagesize,headsepline,bibtotoc,titlepage]{scrartcl}

%\usepackage[ngerman]{babel}		% deutsche Trennmuster
\usepackage[utf8]{inputenc}		% direkte Eingabe von Umlauten & Co. (Vorsicht: Encoding im Editor muss auch UTF-8 sein!)

\usepackage[T1]{fontenc}			% T1-Schriften

\usepackage{mathptmx}			% Times/Mathe \rmdefault
\usepackage[scaled=.90]{helvet}	% Skalierte Helvetica \sfdefault
\usepackage{courier}			% Courier \ttdefault
\usepackage[cache=false,outputdir=.texpadtmp]{minted}
% Zusatzpakete für mehr mathematische Symbole, Einfügen von Grafiken 
% und bessere Bildunterschriften
\usepackage{amsmath,amsthm,amsfonts,graphicx,caption}

% Wenn man direkt mit dem pdflatex eine PDF-Datei erzeugt, sollten diese beiden Pakete eingebunden werden
\usepackage{hyperref} % Hyperlinks anklickbar
\usepackage{ae,aecompl} % bessere Bildschirmschriftarten usw.

\pagestyle{headings}

% Abstand der Kopfzeile vom Text:
\headsep4mm

\typearea[current]{current}     % Satzspiegel neu berechnen

% andere Bildunterschrift mit Hilfe von caption
%\renewcommand{\figurename}{Abb.}
\renewcommand{\captionlabelfont}{\bf}

\title{
	\includegraphics*[width=0.4\textwidth]{hpi_logo.png}\\
	\vspace{24pt}
	 Seaside - An Object-Oriented Web Application Development Framework
}
\subtitle{
	Seminar\\
	Weiterführende Themen zu Internet- und WWW-Technologien\\
	Sommersemester 2016
}
\author{
	Lennard Wolf\\[12pt]
	Betreuer:\\
	Matthias Bauer, Haojin Yang\\
	Prof. Dr. Christoph Meinel
}
\date{\today}

\begin{document}
\maketitle
\tableofcontents
\newpage

\section{Introduction}
Seaside is an open source object-oriented web application development for the Smalltalk programming language. IT DOES THIS AND THAT.... This text aims at a basic understanding of its history and the context in which it was conceived, its concepts, how they differ from other frameworks of its kind, and what working with it looks like. 

This work is structured as follows. Section \ref{sec:history} introduces 
... 
 and Section \ref{sec:conclusion} concludes this work.


\section{History of Seaside}
\label{sec:history}
To understand Seaside's origins, knowledge of the technologies currently popular at the beginnings of its development is required first. This Section will thus start out with such an overview, then present the original motivations behind Seaside, followed by a list of the people involved, a development timeline, and finally the current state of both the feature list and the development in general will be examined.

\subsection{Historical Context}
\label{sec:context} 

\begin{itemize}
\item Unknown: Server side or client side?
\item Javascript was blocked by default Web Apps thus had to be server side
\item User-readable URLs were not a must
\item Ajax did not exist until 2005
\end{itemize}

\subsection{Motivation}
\label{sec:motivation} 

Over the last few years, some best practices have come to be widely accepted in the web development world. Share as little state as possible. Use clean, carefully chosen, and meaningful URLs. Use templates to separate your model from your presentation.
Seaside is a web application framework for Smalltalk that breaks all of these rules and then some. Think of it as an experiment in tradeoffs: if you reject the conventional wisdoms of web development, what benefits can you get in return? Quite a lot, it turns out, and this "experiment" has gained a large open source following, seen years of production use, and been heralded by some as the future of web applications.\footnote{Taken from "Web Heresies: The Seaside Framework' Session notes, OSCON 2006". \url{http://web.archive.org/web/20140830152243id\_/http://conferences.oreillynet.com/cs/os2006/view/e_sess/8942}, accessed: 2016-09-12}


\begin{itemize}
\item Avi Bryant wanted to build interactive Squeak Wiki
\item Inspired by Apple’s WebObjects (Objective-C)
\end{itemize}

\subsection{Development}
\label{sec:dev} 


\begin{itemize}
\item First Release 0.9 in February 2002 by Avi Bryant \& Julian Fitzell, Open Source with MIT License
\item Quick development, version 2.0 came October 2003
\item datepicker was very modern
\end{itemize}

\subsection{Current State}
\label{sec:current} 

\begin{itemize}
\item list of current features:
\item Live debugging and code changing while user interacts with web app
\item debugger in browser
\item no HTML, render page as Morph -> no browser needed
\item CSS, Ajax, Comet, \& Javascript support
\end{itemize}

\section{Object-Oriented Web Application Development}

In this Section we will shortly introduce the object-oriented programming language Smalltalk in which Seaside is written and used. Then we will discuss the differences between more common approaches to web application development and the object-oriented way.

\subsection{Smalltalk}

Smalltalk was created in the 70's at Xerox PARC by Alan Kay \emph{et al.}. It is dynamically typed and purely object-oriented, meaning that \emph{everything in a Smalltalk environment is an object} and can thus be easily interacted with. Smalltalk environments such as the free and open source Squeak or the proprietary Gemstone\footnote{Squeak: \url{http://squeak.org} Gemstone: \url{https://gemtalksystems.com}, both accessed: 2016-09-11} are normally shipped with the Smalltalk core classes (e.g. \mintinline{smalltalk}{Object}) as well as additional ones, depending on the environment's use cases. One of their most important features is \emph{reflection}, with which the programmer is able to inspect and modify every part of the system at runtime. This is especially useful in cases where a system may never shut down, because the changed code does not need to be recompiled but comes into effect immediately. 

The syntax can be described as minimalistic. Smalltalk code consists of sequences of messages sent to objects (\mintinline{smalltalk}{anObject message})\footnote{A good overview on the Smalltalk syntax was created by Chris Rathman and can be found at \url{http://www.angelfire.com/tx4/cus/notes/smalltalk.html}}. Listing \ref{lst:smalltalksyntax} presents syntax basics that are helpful when approaching Seaside.  


\begin{listing}[]%ht]
\begin{minted}{smalltalk}

"Send the message 'not' to 'true' and assign the returned value to a."
a := true not.

"Messages can be chained."
b := a not not not.

"Blocks are sequences of executable code that return the value of the 
last line. They can take any number of arguments (here x and y)."
aBlock := [ :x :y | z := x * y. z - (2 * x) ]

"aBlock is executed with varying parameters depending on b's state."
answer := b ifTrue: [aBlock value: 7 value: 8] 
            iFalse: [aBlock value: 0 value: 0].

\end{minted}
\caption{The core syntax of Smalltalk}
\label{lst:smalltalksyntax}
\end{listing}

\subsection{Comparison with other technologies}

\begin{itemize}
\item comparison of features:
\item everything is an object
\item synchronous vs async
\end{itemize}


\section{Working with Seaside}
\label{sec:workflow}


\begin{itemize}
\item Session for each user
\item ...
\item debugging
\end{itemize}

\begin{figure}[hbp]
\begin{center}
\includegraphics*[width=0.75\textwidth]{images/toolbar.png}\\
\caption{something something}
\label{abb:test}
\end{center}
\end{figure}

\section{Evaluation}
\label{sec:evaluation}

\begin{itemize}
\item Each session object can be multiple megabytes big If server crashes, all sessions are lost
\item Gemstone distributed but expensive
\item Javascript not really debuggable
\item Very hard to implement human readable URLs
\item Not on par with today’s demands of a fast dynamic webpage since its server side
\end{itemize}

\section{Conclusion}
\label{sec:conclusion}



\newpage
\begin{thebibliography}{1}

\bibitem{willems2008}
C.~Willems and C.~Meinel.
``Tele-Lab IT-Security: an Architecture for an online virtual IT Security Lab'',
\emph{International Journal of Online Engineering (iJOE)},
X, 2008.

\end{thebibliography}
\end{document}