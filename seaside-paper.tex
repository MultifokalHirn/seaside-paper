% Koma-Script Basisklasse
\documentclass[a4paper,12pt,pagesize,headsepline,bibtotoc,titlepage]{scrartcl}

%\usepackage[ngerman]{babel}		% deutsche Trennmuster
\usepackage[utf8]{inputenc}		% direkte Eingabe von Umlauten & Co. (Vorsicht: Encoding im Editor muss auch UTF-8 sein!)

\usepackage[T1]{fontenc}			% T1-Schriften

\usepackage{mathptmx}			% Times/Mathe \rmdefault
\usepackage[scaled=.90]{helvet}	% Skalierte Helvetica \sfdefault
\usepackage{courier}			% Courier \ttdefault

% Zusatzpakete für mehr mathematische Symbole, Einfügen von Grafiken 
% und bessere Bildunterschriften
\usepackage{amsmath,amsthm,amsfonts,graphicx,caption}

% Wenn man direkt mit dem pdflatex eine PDF-Datei erzeugt, sollten diese beiden Pakete eingebunden werden
\usepackage{hyperref} % Hyperlinks anklickbar
\usepackage{ae,aecompl} % bessere Bildschirmschriftarten usw.

\pagestyle{headings}

% Abstand der Kopfzeile vom Text:
\headsep4mm

\typearea[current]{current}     % Satzspiegel neu berechnen

% andere Bildunterschrift mit Hilfe von caption
%\renewcommand{\figurename}{Abb.}
\renewcommand{\captionlabelfont}{\bf}

\title{
	\includegraphics*[width=0.4\textwidth]{hpi_logo.png}\\
	\vspace{24pt}
	 Seaside - An Object-Oriented Web Application Development Framework
}
\subtitle{
	Seminar\\
	Weiterführende Themen zu Internet- und WWW-Technologien\\
	Sommersemester 2016
}
\author{
	Lennard Wolf\\[12pt]
	Betreuer:\\
	Matthias Bauer, Haojin Yang\\
	Prof. Dr. Christoph Meinel
}
\date{\today}

\begin{document}
\maketitle
\tableofcontents
\newpage

\section{Introduction}
Seaside is an open source object-oriented web application development for the Smalltalk programming language. IT DOES THIS AND THAT.... This text aims at a basic understanding of its history and the context in which it was conceived, its concepts, how they differ from other frameworks of its kind, and what working with it looks like. 

This work is structured as follows. Section \ref{sec:history} introduces 
... 
 and Section \ref{sec:conclusion} concludes this work.


\section{History of Seaside}
\label{sec:history}
To understand Seaside's origins, knowledge of the technologies currently popular at the beginnings of its development is required first. This Section will thus start out with such an overview, then present the original motivations behind Seaside, followed by a list of the people involved, a development timeline, and finally the current state of both the feature list and the development in general will be examined.

\subsection{Historical Context}
\label{sec:context} 

\begin{itemize}
\item Unknown: Server side or client side?
\item Javascript was blocked by default Web Apps thus had to be server side
\item User-readable URLs were not a must
\item Ajax did not exist until 2005
\end{itemize}

\subsection{Motivation}
\label{sec:motivation} 

\begin{itemize}
\item introduce smalltalk
\item Avi Bryant wanted to build interactive Squeak Wiki
\item Inspired by Apple’s WebObjects (Objective-C)
\end{itemize}

\subsection{Development}
\label{sec:dev} 


\begin{itemize}
\item First Release 0.9 in February 2002 by Avi Bryant \& Julian Fitzell, Open Source with MIT License
\item Quick development, version 2.0 came October 2003
\item datepicker was very modern
\end{itemize}

\subsection{Current State}
\label{sec:current} 

\begin{itemize}
\item list of current features:
\item Live debugging and code changing while user interacts with web app
\item debugger in browser
\item no HTML, render page as Morph -> no browser needed
\item CSS, Ajax, Comet, \& Javascript support
\end{itemize}

\section{Object-Oriented Web Application Development}


\begin{itemize}
\item list of current features
\end{itemize}


\section{Working with Seaside}
\label{sec:workflow}


\begin{itemize}
\item Session for each user
\end{itemize}

\begin{figure}[hbp]
\begin{center}
\includegraphics*[width=0.75\textwidth]{images/toolbar.png}\\
\caption{something something}
\label{abb:test}
\end{center}
\end{figure}

\section{Evaluation}
\label{sec:evaluation}

\begin{itemize}
\item Each session object can be multiple megabytes big If server crashes, all sessions are lost
\item Gemstone distributed but expensive
\item Javascript not really debuggable
\item Very hard to implement human readable URLs
\item Not on par with today’s demands of a fast dynamic webpage since its server side
\end{itemize}

\section{Conclusion}
\label{sec:conclusion}



\newpage
\begin{thebibliography}{1}

\bibitem{willems2008}
C.~Willems and C.~Meinel.
``Tele-Lab IT-Security: an Architecture for an online virtual IT Security Lab'',
\emph{International Journal of Online Engineering (iJOE)},
X, 2008.

\end{thebibliography}
\end{document}